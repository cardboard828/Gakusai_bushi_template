%-----グラフィック------------------------------------------------------
%呪文
\usepackage[dvipdfmx]{graphicx, xcolor}
\definecolor{mygreen}{rgb}{0.1,0.35,0.35}
\definecolor{mygreen2}{rgb}{0.1,0.4,0.3}
\definecolor{myblue}{rgb}{0,0.25,0.9}
\definecolor{myred}{rgb}{1,0.2,0.2}
%\usepackage[dvipsnames]{xcolor}
%枠つき部分作るならtcolorbox一択、使い方はググればわかる
\usepackage{tcolorbox}
%tcolorboxの機能拡張。ページまたぎを実現/ボックスの装飾機能の拡張
\tcbuselibrary{breakable,skins,raster}
%tcolorboxの使用例。ゼミ資料のグレーの枠はこう定義して使ってる。
\newtcolorbox{mybox}{colframe=gray!10, colback=gray!10, sharp corners}
%記事タイトル用
\newtcolorbox{TTL}{colframe=white, colback=white}

%見出しの体裁のカスタマイズ
\usepackage[explicit]{titlesec}
%下を入れるとpartが見えなくなる
%\titleformat{\part}[hang]{\bfseries\gtfamily}{}{10pt}{}
%chapter(タイトル、名前)と本文の距離の設定
\titlespacing{\chapter}{0.01zw}{1zw}{4zw}[0.01zw]

%ヘッダー、フッターの設定
\usepackage{fancyhdr}
\pagestyle{fancy}
\fancyhead{}
\renewcommand{\sectionmark}[1]{\markboth{\thesection #1}{}}
\renewcommand{\subsectionmark}[1]{\markright{\thesubsection #1}}
\fancyhead{\gdef\headrulewidth{.0pt}}
\fancyhead[ER]{\leftmark}
\fancyhead[EL]{\thepage}
\fancyhead[OL]{\rightmark}
\fancyhead[OR]{\thepage}
\cfoot{}


%thebibliography環境をchapterではなくsectionみたいに扱う
\usepackage{etoolbox}
\patchcmd{\thebibliography}{\chapter*}{\section*}{}{}

%sectionの番号をchapter-independentにする
\usepackage{remreset}
\makeatletter
  \@removefromreset{section}{chapter}
\makeatother
\renewcommand{\thesection}{\arabic{section}}

\setcounter{secnumdepth}{3}

%最悪なんだけど、何用なのか忘れてしまった
\usepackage[pass]{geometry}

%奥付け
\usepackage{tabularx}
\usepackage[bb=boondox,bbscaled=.95,cal=boondoxo]{mathalfa}
\renewcommand{\hrulefill}{%
  \leavevmode\leaders\hrule height 1pt\hfill\kern0pt }

%-----以下数式関係----------------------------------------------------

% 数式, 米国数学会が開発したのがamsmath, フォントを使うためのパッケージがamssymbs
\usepackage{amsmath,amssymb}
%定理環境、コピペ
\usepackage{amsthm}
\theoremstyle{definition}
\newtheorem{dfn}{定義}[section]
\newtheorem{prop}{命題}[section]
\newtheorem{lem}{補題}[section]
\newtheorem{thm}{定理}[section]
\newtheorem*{thm*}{定理}
\newtheorem{cor}{系}[section]
\newtheorem{rem}{注意}[section]
\newtheorem*{rem*}{注意}
\newtheorem{fact}{事実}[section]
\newtheorem{e.g.}{例}[section]

%証明終わりの四角
\renewcommand{\qedsymbol}{$\square$}

%花文字?(2023/6/10追加), RSFS, Ralph Smith's Formal Script(2024/3/3追記)
\usepackage{mathrsfs}

% 数式太文字
\usepackage{bm}

%proofカスタム, 適宜変更
\renewcommand{\proofname}{\textbf{証明}}

%\label{}して\ref, \eqref で参照した数式のみに数式番号が振られるようにしてる
%だからequationやalignを*なしで使っても参照しない限り式番号はつかない
\usepackage{mathtools}
\mathtoolsset{showonlyrefs=true}
%2024/10/3追加(工藤英哲さん用のパッケージ)
\usepackage{empheq}

%白紙ページ生成用
\usepackage{afterpage}

\newcommand\blankpage{%
    \null
    \thispagestyle{empty}%
    \addtocounter{page}{0}%
    \newpage}


%-----以下自然科学関係------------------------------------------------------

%便利なので
\usepackage{physics}
%単位
\usepackage{siunitx}
%元素とか, 化学反応式とか
\usepackage[version=4]{mhchem}
\usepackage{chemfig}

%-----以下図表の配置関係------------------------------------------------------

% 表の強制固定
\usepackage{float}

%-----以下その他------------------------------------------------------

% comment、\begin{comment}~\end{comment}で挟んだ箇所がコンパイル時に無視される
\usepackage{comment}
%OTFパッケージ・漢字用、例えば「﨑」はuplatexだと出力できないけど、対応するjisコードを\UTF{}で読み込むと出力できる
\usepackage[jis2004, deluxe]{otf}
%\url用
%\usepackage{url}

\usepackage[
dvipdfmx,% 欧文ではコメントアウトする
setpagesize=false,%
bookmarks=true,%
bookmarksdepth=tocdepth,%
bookmarksnumbered=true,%
hypertexnames=false,% 
linktoc=chapter,%
colorlinks=false,%
allcolors=blue%
pdftitle={},%
pdfsubject={},%
pdfauthor={},%
pdfkeywords={}%
]{hyperref}
% PDFのしおり機能の日本語文字化けを防ぐ((u)pLaTeXのときのみかく)
\usepackage{pxjahyper}
\hypersetup{
    colorlinks=true, %ここをfalseにすればlinkの文字に色がつかずに四角で囲まれるようになる
    citecolor=myred,
    linkcolor=myblue,
    urlcolor=mygreen2,%teal,
}