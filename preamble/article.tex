
%見出しの体裁のカスタマイズ用
\usepackage[explicit]{titlesec}

%記事タイトル用の勝手なtcolorboxを定義
\newtcolorbox{TTL}{colframe=white, colback=white}

%chapter(タイトル、名前)と本文の距離の設定
\titlespacing{\chapter}{0.01zw}{1zw}{4zw}[0.01zw]

%各記事のタイトルをchapterで作成することでTOCを表示すれば記事名が目次に載るようにしている
%ただ,記事のタイトルが普通のchapterのようになってはオカシイので,titlesecを使ってchapterの表示のさせ方を,articleのタイトルのようにしている。
%ただタイトルには名前もつけたい,しかしタイトルの後に改行した名前をくっつけるとTOCで表示した時に変な感じになる
%そこで各記事に対して,chapter名を真ん中に表示させて,その後筆者の名前をくっつける,という装飾したchapterを各記事について作って解決した。
%以下がタイトルの書き方の例(mcfamilyにしてるのはarticleのタイトルに寄せるため)
%*をつけているのは章番号は不要なため
%\addcontentsline の行とその上の行は*をしたがTOCにはタイトルを反映させるため

%~~~~~~~~~~~~~~~~~~~~~~~~~~~~~~~~~~~~~~~~~~~~~~~~~~~~~~~
% \titleformat{\chapter}[hang]{\mcfamily}{}{0pt}{
%   \begin{TTL}
%     \centering
%     {\LARGE #1}
% \end{TTL}
% }[\centerline{理学部物理学科 アインシュタイン}]

% \phantomsection
% \addcontentsline{toc}{chapter}{相対性理論}
% \chapter*{相対性理論}
%\input{記事へのパス}
%~~~~~~~~~~~~~~~~~~~~~~~~~~~~~~~~~~~~~~~~~~~~~~~~~~~~~~~

%thebibliography環境をchapterではなくsectionみたいに扱う
\usepackage{etoolbox}
\patchcmd{\thebibliography}{\chapter*}{\section*}{}{}

%sectionの番号をchapter-independentにする
\usepackage{remreset}
\makeatletter
  \@removefromreset{section}{chapter}
\makeatother
\renewcommand{\thesection}{\arabic{section}}

%subsubsectionまで使えるようにする
\setcounter{secnumdepth}{3}


